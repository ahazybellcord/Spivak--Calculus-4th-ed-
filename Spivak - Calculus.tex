\documentclass[12pt]{article}
%\topmargin=-0.3in
%\leftmargin=0.5in
%\rightmargin=0.5in
\headsep=0.25in
\parindent=0px
\setlength{\parskip}{0.5em}

\usepackage[margin=1in]{geometry}
\usepackage{amssymb}
\usepackage{amsmath}
\usepackage{enumitem}


\newcommand{\qed}{\blacksquare}
\newcommand{\then}{\Rightarrow}
\newcommand{\abs}[1]{|#1|}
\renewcommand{\labelenumii}{(\roman{enumii})}
   

\begin{document}
\textbf{PROBLEM SOLUTIONS}\\

\underline{Chapter 1: Basic Properties of Numbers}
\begin{enumerate}

\item Prove the following:
\begin{enumerate}
% Problem 1
\item If $ax = a$ for some number $a \neq 0$, then $x = 1$.

\underline{Proof}: If $a \neq 0$ then there exists $a^{-1}$ such that 
\begin{align} 
a \cdot a^{-1} &= a^{-1} \cdot a = 1 \tag{P7}
\end{align}
Then:
\begin{align}
a^{-1}\cdot (a\cdot x) &= (a^{-1}\cdot a) \cdot x \tag{P5}\\
			     &= 1 \cdot x \tag{P7}\\
			     &= x \tag{P6}
\end{align}
and
\begin{align}
a^{-1}\cdot a &= 1 \tag{P7}
\end{align}
Then $ax = a \Rightarrow x = 1$ $\qed$

\item $x^2 - y^2 = (x - y)(x + y).$
\begin{align}
\text{\underline{Proof}: } (x - y)(x + y) &= x \cdot (x + y) + (-y) \cdot (x + y) \tag{P9}\\
		    &= x \cdot x + x \cdot y + (-y) \cdot x + (-y) \cdot y \tag{P9}\\
		    &= x \cdot x + x \cdot y + x \cdot (-y) + (-y) \cdot y \tag{P4}\\
		    &= x \cdot x + x \cdot (y + (-y)) + (-y) \cdot y \tag{P9}\\
		    &= x \cdot x + x \cdot 0 + (-y) \cdot y \tag{P3}\\
		    &= x \cdot x + (-y) \cdot y \tag{P9}\\
		    &= x \cdot x + -(y \cdot y) \tag{P5}\\
		    &= x^2 - y^2 \hspace{.1cm} \blacksquare \notag 	    
\end{align}

\item If $x^2 = y^2$, then $x = y$ or $x = -y$.

\underline{Proof}: Since $x^2 = y^2$ we have
\begin{align}
x^2 - y^2 &= y^2 - y^2 \notag \\
	       &= 0 \tag{P3}
\end{align}
and $x^2 - y^2 = (x - y)(x + y)$ by 1(ii). Therefore
\begin{align}
(x - y)(x + y) &= 0 \notag
\end{align}
Then if $x \neq -y$
\begin{align}
((x - y) (x + y)) \cdot (x + y)^{-1} &= (x - y) \cdot ((x + y) (x + y)^{-1}) \tag{P5}\\
						         &= (x - y) \cdot 1 \tag{P7}\\
					                  &= x - y \tag{P6}
\end{align}
and since
\begin{align}
0 \cdot (x + y)^{-1} &= 0 \tag{P9}
\end{align}
we have $x - y = 0 \Rightarrow x = y$ (P3). $\blacksquare$ \\
If we start with $x \neq y$ a similar proof will show that $x = -y$.

\item $x^3 - y^3 = (x - y)(x^2 + xy + y^2)$.
\begin{align}
\text{\underline{Proof}: } (x - y)(x^2 + xy + y^2) &= (x - y) \cdot x^2 + (x - y) \cdot xy + (x - y) \cdot y^2  \tag{P9}\\
									&= x \cdot x^2 - y \cdot x^2 + x \cdot xy - y \cdot xy + x \cdot y^2 - y \cdot y^2 \tag{P9}\\
									&= x^3 - yx^2 + x^2y + yxy + xy^2 - y^3 \notag \\
									&= x^3 - x^2y + x^2y + xy^2 - xy^2 - y^3 \tag{P5}\\
									&= x^3 - y^3 \tag{P3}
\end{align}

\item $x^n - y^n = (x - y)(x^{n-1} + x^{n-2}y + \dots + xy^{n-2} + y^{n-1})$.

$\text{\underline{Proof}: }(x - y)(x^{n-1} + x^{n-2}y + \dots + xy^{n-2} + y^{n-1}) = $
\begin{multline*}
= x \cdot x^{n-1} + x \cdot x^{n-2}y + \dots + x \cdot xy^{n-2} + x \cdot y^{n-1} \\- y \cdot x^{n-1} - y \cdot x^{n-2}y - \dots - y \cdot xy^{n-2} - y \cdot y^{n-1}
\end{multline*}
\begin{multline*}
= x^n + x^{n-1}y + \dots + x^2y^{n-2} + xy^{n-1} \\- x^{n-1} \cdot y - x^{n-2}y \cdot y- \dots - xy^{n-2} \cdot y - y^n \hspace{.2cm} \text{(P5)}
\end{multline*}
\begin{multline*}
= x^n + x^{n-1}y + \dots + x^2y^{n-2} + xy^{n-1} - x^{n-1}y - x^{n-2}y^2- \dots - xy^{n-1} - y^n
\end{multline*}
\begin{align*}
&= x^n + x^{n-1}y - x^{n-1}y + \dots + xy^{n-1} - xy^{n-1} - y^n \tag{P1}\\
&= x^n + 0 + \dots + 0 - y^n \hspace{.2cm} \tag{P3}\\
&= x^n - y^n \hspace{.1cm} \blacksquare \tag{P2}\\
\end{align*}
							
\item $x^3 + y^3 = (x + y)(x^2 - xy + y^2)$.

Using the result of 1(iv) with $-y$ in place of $y$, we have
\begin{align*}
x^3 - (-y)^3 = x^3 - -y^3 = x^3 + y^3 &= (x - (-y))(x^2 + x(-y) + (-y)^2) \\
							&= (x + y)(x^2 - xy + y^2) \hspace{.1cm} \blacksquare
\end{align*}
\end{enumerate}
% Problem 2
\item What is wrong with the following ``proof"? Let $x = y$. Then
\begin{align*}
x^2 &= xy,\\
x^2 - y^2 &= xy - y^2,\\
(x + y)(x - y) &= y(x - y),\\
x + y &= y,\\
2y &= y,\\
2 &= 1.
\end{align*}
\underline{Answer}: Between lines 3 and 4 we perform a division by $x - y$. But since $x = y$, this means we divided by 0, which is undefined.\\
% Problem 3
\item Prove the following:
\begin{enumerate}
\item $\frac{a}{b}=\frac{ac}{bc}$, if $b,c \neq 0$.

\underline{Proof}: $\frac{a}{b} = a \cdot b^{-1}$ and $\frac{ac}{bc} = ac \cdot (bc)^{-1}$\\ 
Thus $$ac \cdot (bc)^{-1} \cdot bc = ac$$ and $$a \cdot b^{-1} \cdot bc = a \cdot (b^{-1}b) \cdot c = ac$$ and therefore $$\frac{a}{b}=\frac{ac}{bc} \hspace{.1cm} \blacksquare$$

\item $\frac{a}{b} + \frac{c}{d} =\frac{ad + bc}{bd}$ if $b,d \neq 0$.

\underline{Proof}: Since $b,d \neq 0$ we can multiply each side by $bd$.
\begin{align*}
\frac{ad + bc}{bd} \cdot bd &= ((ad + bc) \cdot (bd)^{-1}) \cdot bd\\
			   &= (ad + bc) \cdot ((bd)^{-1} \cdot bd)\\
			   &=	 ad + bc\\
\end{align*}
and	
\begin{align*}
\left(\frac{a}{b} + \frac{c}{d}\right) \cdot bd &= (a \cdot b^{-1} + c \cdot d^{-1}) \cdot bd\\
			   &=	 ab^{-1} \cdot bd + cd^{-1} \cdot db\\
			   &=  a(b^{-1}b)d + c(d^{-1}d)b\\
			   &= ad + cb\\
			   &= ad + bc
\end{align*}
Therefore 	$\frac{a}{b} + \frac{c}{d} =\frac{ad + bc}{bd}$ $\blacksquare$

\item $(ab)^{-1} = a^{-1}b^{-1}$, if $a,b \neq 0$.

\underline{Proof}: Since $a,b \neq 0$ we can multiply both sides by $ba$.
\begin{align*}
a^{-1}b^{-1} \cdot ba &= a^{-1}(b^{-1} \cdot b)a\\
				&= a^{-1} \cdot a\\
				&= 1
\end{align*}
Since 
\begin{align*}
(ba)^{-1} \cdot ba = 1 &\Rightarrow a^{-1}b^{-1} \cdot ba = (ba)^{-1} \cdot ba \\
				 &\Rightarrow a^{-1}b^{-1} = (ba)^{-1}  = (ab)^{-1} \hspace{.1cm} \blacksquare
\end{align*}

\item $\frac{a}{b} \cdot \frac{c}{d} = \frac{ac}{db}$, if $b,d \neq 0$.

\underline{Proof}:
$$\frac{a}{b} \cdot \frac{c}{d} = ab^{-1} \cdot cd^{-1} = acd^{-1}b^{-1} = ac(db)^{-1} = \frac{ac}{db} \hspace{.1cm} \blacksquare$$

\item $\frac{a}{b} \diagup \frac{c}{d} = \frac{ad}{bc}$ if $b,c,d \neq 0$.

\underline{Proof}:
$$ \frac{ad}{bc} \cdot \frac{c}{d} = ad \cdot (bc)^{-1} \cdot cd^{-1} = ad \cdot b^{-1}c^{-1} \cdot cd^{-1} = ab^{-1}cc^{-1}dd^{-1} = ab^{-1} = \frac{a}{b}$$ therefore $$\frac{ad}{bc} = \frac{a}{b} \cdot (cd^{-1})^{-1} = \frac{a}{b} \diagup \frac{c}{d}$$

\item If $b,d \neq 0$, then $\frac{a}{b} = \frac{c}{d}$ if and only if $ad = bc$. Also determine when $\frac{a}{b} = \frac{b}{a}$.

\underline{Proof:}
\begin{align*} \frac{a}{b} = \frac{c}{d} &\Rightarrow \frac{a}{b} \cdot bd = \frac{c}{d} \cdot bd \\
						        &\Rightarrow ab^{-1}bd = cd^{-1}db \\
						        &\Rightarrow ad = cb = bc.
\end{align*}
\begin{align*} 
ad = bc &\Rightarrow ad \cdot d^{-1}b^{-1} = bc \cdot d^{-1}b^{-1} \\
             &\Rightarrow add^{-1}b^{-1} = cbb^{-1}d^{-1} \\
             &\Rightarrow ab^{-1} = cd^{-1} \\
             &\Rightarrow \frac{a}{b} = \frac{c}{d} \hspace{.1cm} \blacksquare
\end{align*}
\end{enumerate}
% Problem 4
\item Find all numbers $x$ for which
\begin{enumerate}
\item $4 - x < 3 - 2x$
$$4 - x < 3 - 2x \Rightarrow x < -1$$
Ans: $\{x\in \mathbb{R} : x < -1\}$
\item $5 - x^2 < 8$
$$5 - x^2 < 8 \Rightarrow x^2 > -3 $$
Ans: all $x\in \mathbb{R}$ satisfy this inequality.

\item $5 - x^2 < -2$
$$ 5 - x^2 < -2 \Rightarrow x^2 > 7 \Rightarrow (x > \sqrt{7}) \text{ OR } (x < -\sqrt{7})$$
Ans: $\{x\in \mathbb{R} : x > \sqrt{7}\} \cup \{x\in \mathbb{R} : x < -\sqrt{7}\} $

\item $(x - 1)(x - 3) > 0$
$$ ab > 0 \Rightarrow (a > 0, b > 0) \text{ OR } (a < 0, b < 0)$$
Case 1: $a > 0, b > 0$:
$$ x - 1 > 0 \Rightarrow x > 1$$
$$ x - 3 > 0 \Rightarrow x > 3$$
$\{x\in \mathbb{R} : x > 1\} \cap \{x\in \mathbb{R} : x > 3\} = \{x\in \mathbb{R} : x > 3\}$.\\
Case 2: $a < 0, b < 0$:
$$ x - 1 < 0 \Rightarrow x < 1$$
$$ x - 3 < 0 \Rightarrow x < 3$$
Ans: $\{x\in \mathbb{R} : x < 1\} \cap \{x\in \mathbb{R} : x < 3\} = \{x\in \mathbb{R} : x < 1\}$.\\
Therefore the full answer is: $\{x\in \mathbb{R} : x > 3\} \cup \{x\in \mathbb{R} : x < 1\}$.

\item $x^2 - 2x + 2 > 0$
$$ x^2 - 2x + 2 = (x - 1)^2 + 1 > 0 \Rightarrow (x - 1)^2 > -1 $$
If $(x - 1)$ is positive or negative, $(x - 1)^2 > 0 > -1$. \\If it is zero, then clearly $(x - 1)^2 > -1$.\\
Ans: all $x\in \mathbb{R}$ satisfy this inequality.

\item $x^2 + x + 1 > 2$
$$ x^2 + x + 1 > 2 \Rightarrow x^2 + x - 1 > 0$$
Employing the quadratic formula: $$ x > \frac{1 \pm \sqrt{5}}{2}$$
Ans: $\{x\in \mathbb{R} : x > \frac{1 \pm \sqrt{5}}{2}\}$

\item $x^2 - x + 10 > 16$
$$x^2 - x + 10 > 16 \Rightarrow x^2 - x - 6 > 0 \Rightarrow (x -3)(x + 2) > 0$$
Then if $(x - 3) > 0, (x + 2) > 0 \Rightarrow x > 3$.\\ 
If $(x - 3) < 0, (x + 2) < 0 \Rightarrow x < -2$.\\
Ans: $\{x\in \mathbb{R} : x > 3\} \cup \{x\in \mathbb{R} : x < -2\}$

\item $x^2 + x + 1 > 0$

Ans: all $x\in \mathbb{R}$.

\item $(x - \pi)(x + 5)(x - 3) > 0$

Either all three terms are positive, else two are negative and one is positive.\\
Case 1: 
$(x - \pi) > 0, (x + 5) > 0, (x - 3) > 0$\\
$x > \pi$ and $x > -5$ and $x > 3 \Rightarrow x > \pi$.\\
Case 2:
$(x - \pi) > 0, (x + 5) < 0, (x - 3) < 0$\\
$x > \pi$ and $x < -5$ and $x < 3 \Rightarrow x = \emptyset$\\
Case 3:
$(x - \pi) < 0, (x + 5) > 0, (x - 3) < 0$\\
$x < \pi$ and $x > -5$ and $x < 3 \Rightarrow x > -5$ and $x < 3$.\\
Case 4:
$(x - \pi) < 0, (x + 5) < 0, (x - 3) > 0$\\
$x < \pi$ and $x < -5$ and $x > 3 \Rightarrow x = \emptyset$.\\
Ans: $\{x :  x > \pi\} \cup \{x : -5 < x < 3\}$

\item $(x - \sqrt[3]{2})(x - \sqrt{2}) > 0$

Case 1: $(x - \sqrt[3]{2}) > 0, (x - \sqrt{2}) > 0$\\
$x > \sqrt[3]{2}$ and $x > \sqrt{2} \Rightarrow x > \sqrt{2}$\\
Case 2: $(x - \sqrt[3]{2}) < 0, (x - \sqrt{2}) < 0$\\
$x < \sqrt[3]{2}$ and $x < \sqrt{2} \Rightarrow x <  \sqrt[3]{2}$\\
Ans: $\{x :  x > \sqrt{2}\} \cup \{x : x < \sqrt[3]{2}\}$

\item $2^x < 8$
$$2^x < 8 \Rightarrow x < \text{log}_2{8} = 3$$
Ans: $\{x : x < 3\}$

\item $x + 3^x < 4$

Take $x = 1$. Then $$x + 3^x = 1 + 3 = 4 \nless 4.$$\\
Now take $x > 1$. Then $$x + 3^x > 1 + 3 = 4 \nless 4.$$\\
Ans: $\{x : x < 1\}$

\item $\frac{1}{x} + \frac{1}{1 - x} > 0$
$$\frac{1}{x} + \frac{1}{1 - x} > 0 \Rightarrow \frac{1}{x} > -\frac{1}{1 - x} \Rightarrow 1 - x > -x \Rightarrow 1 > 0$$
Since this is a contradiction, there does not exist an $x$ such that $\frac{1}{x} + \frac{1}{1 - x} > 0$.\\
Ans: $\emptyset$

\item $\frac{x - 1}{x + 1} > 0$

Due to the presence of $x + 1$ in the denominator, we are limited to $x \neq -1$. This permits us to multiply both sides of the inequality by $x + 1$.\\
$$\frac{x - 1}{x + 1} > 0 \Rightarrow x - 1 > 0 \Rightarrow x > 1$$
Ans: $\{x : x > 1\}$
\end{enumerate}
% Problem 5
\item Prove the following:
\begin{enumerate}
\item If $a < b$ and $c < d$, then $a + c < b + d$.
$$ a < b, c < d \Rightarrow 0 < b - a, 0 < d - c \Rightarrow 0 < (b - a) + (d - c) \Rightarrow a + c < b + d $$

\item If $a < b$, then $-b< -a$.
$$ a < b \Rightarrow 0 < b - a \Rightarrow -b < -a$$

\item If $a < b$ and $c > d$, then $a - c < b - d$.
$$ a < b, c > d \Rightarrow 0 < b - a, 0 < c - d \Rightarrow 0 < (b - a) + (c - d) \Rightarrow a - c < b - d$$

\item If $a < b$ and $c > 0$, then $ac < bc$.
$$ a < b \Rightarrow 0 < b - a \Rightarrow 0 < c(b - a) = cb - ca = bc - ac \Rightarrow ac < bc$$

\item If $a < b$ and $c < 0$, then $ac > bc$.
$$ c < 0 \Rightarrow -c > 0$$
$$ a < b \Rightarrow 0 < b - a \Rightarrow 0 < -c(b - a) \Rightarrow 0 < -cb + ca = -bc + ac \Rightarrow bc < ac$$

\item If $a > 1$, then $a^2 > a$.
$$ a > 1 \Rightarrow a > 0$$
$$ a > 1 \Rightarrow a - 1 > 0 \Rightarrow a(a - 1) = a^2 - a > 0 \Rightarrow a^2 > a$$

\item If $0 < a < 1$, then $a^2 < a$.

$$0 < a < 1 \Rightarrow a > 0, a - 1 < 0 \Rightarrow a(a-1) = a^2 - a < 0 \Rightarrow a^2 < a$$

\item If $0 \leq a < b$ and $0 \leq c < d$, then $ac < bd$.\\

If $a = 0$ or $c = 0$ then $ac = 0$.\\ 
Since $b > a = 0$ and $d > c = 0$, $bd > 0 = ac$.\\
Otherwise, we have $0 < a < b$ and $0 < c < d$.\\
Then $$0 < a < b, c > 0 \Rightarrow 0 < ac < bc$$ and $$0 < c < d, b > 0 \Rightarrow 0 < bc < bd$$
which gives us $$0 < ac < bc < bd \Rightarrow ac < bd$$

\item If $0 \leq a < b$, then $a^2 < b^2$.

If $a = 0$ then $$b > 0 \Rightarrow b^2 > 0 = a^2.$$\\
Otherwise $$0 < a < b \Rightarrow 0 < a^2 < ab \text{ and } 0 < ab < b^2 \Rightarrow 0 < a^2 < ab < b^2 \Rightarrow a^2 < b^2.$$
(Or we could just apply 5(viii) with $c = a$ and $d = b$.)\\

\item If $a,b \geq 0$ and $a^2 < b^2$, then $a < b$.

If $a = 0$ and $a^2 = 0 < b^2$ then $0 \cdot b^{-1} < b^2 \cdot b^{-1} \Rightarrow 0 < b$.\\
Otherwise $a > 0$ and $b > 0$. Then $$a^2 < b^2 \Rightarrow 0 < b^2 - a^2 = (b - a)(b + a).$$
Case 1: $b - a > 0, b + a > 0$.
$$b - a > 0 \Rightarrow b > a \text{ and } b + a > 0 \Rightarrow b > -a.$$ Since $a > 0 \Rightarrow a > -a \Rightarrow b > a $.\\
Case 2: $b - a < 0, b + a < 0$
$$b - a < 0 \Rightarrow b < a \text{ and } b + a < 0 \Rightarrow b < -a.$$
But since $a > 0 \Rightarrow -a < 0 \Rightarrow b < 0$. This is a contradiction. Case 2 never occurs.
\end{enumerate}
% Problem 6
\item \hfill
\begin{enumerate}[label=(\alph*)]
\item Prove that if $0 \leq x < y$, then $x^n < y^n$, $n = 1, 2, 3, \dots$.

\underline{Proof}: We can apply theorem 5(viii) any number of times with $a = c = x$ and $b = d = y$.

\item Prove that if $x < y$ and $n$ is odd, then $x^n < y^n$.

When $n=1$, $x^1 < y^1 \iff x < y$ which is true by hypothesis. Assume the theorem holds for all odd $n$ up to some $k$. That is, $x^n < y^n$ for $n=1,3,5\dots, k$. Then $x^{k+2}<y^{k+2} \iff x^{2}x^{k}<y^{2}y^{k} \iff x^k < (y/x)^2 y^k$. Since $x<y$, we have $y/x > 1$ and so from 5(vi) we know $(y/x)^2 >y/x>1$ and so $y^k < (y/x)^2 y^k$. Since we assume from the inductive step that $x^k < y^k$, we must have $x^k < y^k < (y/x)^2 y^k$.

\item Prove that if $x^n = y^n$ and $n$ is odd, then $x = y$.

Proof by contradiction. Assume $x^n = y^n$ for $n$ odd but $x\neq y$. Take $x<y$. From 6(b), we have that $x^n < y^n$. Similarly, for $y<x$ we have $y^n < x^n$. By contradiction, we must have that $x=y$.

\item Prove that if $x^n = y^n$ and $n$ is even, then $x = y$ or $x = -y$.

Proof by contradiction. Assume $x^n = y^n$ for $n$ even but $x\neq y$ and $x\neq -y$. Then necessarily $x<y$ or $x>y$. Take $x<y$.

Case 1: If $0\leq x<y$ then we have from 6(a) that $x^n<y^n$ for any $n$, a contradiction.

Case 2: Assume $x<0,y\geq 0$. Then we have $0<-x\leq y$ or $0\leq y < -x$. Then from 6(a), either $(-x)^n < y^n$ or $y^n < (-x)^n$, respectively. In either case, we have $(-x)^n\neq y^n$. Since $n$ is even, we can write $n=2k$ for some integer $k$. Then $(-x)^n=((-x)^2)^k=(x^2)^k=x^{2k}=x^n$. Then we find that in either case $x^n\neq y^n$ which contradicts our assumption.

Case 3: Assume $x,y\leq 0$. Then $0\leq -y<-x$. From 6(a) $(-y)^n < (-x)^n$ but since $(-y)^n=y^n$ and $(-x)^n=x^n$ from the argument above, we find that $y^n < x^n$, a contradiction.

Case 4: Assuming $x>0, y\leq 0$ is a contradiction of $x<y$.

The proof for $x>y$ follows from symmetry arguments.
\end{enumerate}
% Problem 7
\item Prove that if $0 < a < b$, then $$a < \sqrt{ab} < \frac{a+b}{2} < b.$$

\underline{Proof}: $$0 < a < b \Rightarrow 0 < a^2 < ab$$
$$0 < a < b \Rightarrow 0 < ab < b^2$$
$$0 < a^2 < ab < b^2 \Rightarrow 0 < a < \sqrt{ab} < b$$
$$\left(\frac{a + b}{2}\right)^2 = \frac{a^2 + 2ab + b^2}{2} = \frac{a^2}{2} + ab + \frac{b^2}{2} > ab \Rightarrow \frac{a + b}{2} > \sqrt{ab}$$
$$a - b < 0 \Rightarrow \frac{a - b}{2} < 0 \Rightarrow \frac{a - b}{2} + b < b \Rightarrow \frac{a - b}{2} + b = \frac{a + b}{2} < b$$

% Problem 8
\item Although the basic properties of inequalities were stated in terms of the collection $P$ of all positive numbers, and $<$ was defined in terms of $P$, this procedure can be reversed. Suppose that P10-P12 are replaced by
\begin{enumerate}[label=(P'\arabic*)]
\setcounter{enumii}{10}
\item For any numbers $a$ and $b$ one, and only one, of the following holds:
\begin{enumerate}[label=(\roman*)]
\item $a=b$,
\item $a<b$,
\item $b<a$.
\end{enumerate}
\item For any numbers $a$, $b$, and $c$, if $a<b$ and $b<c$, then $a<c$.
\item For any numbers $a$, $b$, and $c$, if $a<b$, then $a+c<b+c$.
\item For any numbers $a$, $b$, and $c$, if $a<b$ and $0<c$, then $ac<bc$.
\end{enumerate}
Show that P10-P12 can then be deduced as theorems.

(P10) Applying P'10 with $b=0$ we have that for any number $a$, either $a=0$, $a<0$ or $0<a$. Since $P$ is defined to be the collection of all numbers $a>0$, this is equivalent to the statement of P10.

(P11) Let $0<x,y$ (since using $a$ and $b$ would amount to an abusive confusion of variables in different scopes). Applying P'12 with $a=0, b=x, c=y$, we have:
\begin{align}
		0 + y &< x + y \tag{P'12}\\
             y &< x + y \tag{P2}\\
		0 &< x + y \tag{P'11}
\end{align}
The last step follows from the fact that $0<y$. So we have shown that for numbers $0<x$, $0<y$, we have that $0<x+y$, an equivalent statement to (P11).

(P12) Let $0<x,y$. Applying P'13 with $a=0, b=x, c=y$, we have:
\begin{align}
		0\cdot y &< x\cdot y \tag{P'13}\\
             (x + (-x)) \cdot y &< x\cdot y \tag{P3}\\
		y \cdot (x + (-x)) &< x\cdot y \tag{P4}\\
		y\cdot x + y\cdot (-x) &< x\cdot y \tag{P9}\\
		y\cdot x + (-y\cdot x) &< x\cdot y \tag{P8}\\
		0 &< x\cdot y \tag{P3}\\
\end{align}
Therefore we've shown that when $0<x$ and $0<y$, we have $0<x\cdot y$, an equivalent statement to (P12).

% Problem 9
\item Express each of the following with at least one less pair of absolute value signs.
\begin{enumerate}[label=(\roman*)]
\item $\abs{\sqrt{2}+\sqrt{3}-\sqrt{5}+\sqrt{7}}$

Since $7>5$ and $f(x)=\sqrt{x}$ is monotonically increasing from $x>0$, $\sqrt{7}>\sqrt{5} \then \sqrt{7}-\sqrt{5} > 0$. The total sum is positive so we can drop the absolute value signs and write $\abs{\sqrt{2}+\sqrt{3}-\sqrt{5}+\sqrt{7}}=\sqrt{2}+\sqrt{3}-\sqrt{5}+\sqrt{7}$.

\item $\abs{(\abs{a+b}-\abs{a}-\abs{b})}$

$\abs{a+b}-\abs{a}-\abs{b}=\abs{a+b}-(\abs{a}+\abs{b})$. From the Triangle Inequality, $\abs{a+b}\leq\abs{a}+\abs{b}\then\abs{a+b}-(\abs{a}+\abs{b})\leq 0$. Since $\abs{x}=-x$ for $x\leq 0$, we can write $\abs{(\abs{a+b}-\abs{a}-\abs{b})}=-(\abs{a+b}-\abs{a}-\abs{b})=\abs{a}+\abs{b}-\abs{a+b}$.

\item $\abs{(\abs{a+b}+\abs{c}-\abs{a+b+c})}$

Let $d=a+b$. Then $\abs{a+b}+\abs{c}-\abs{a+b+c}=\abs{d}+\abs{c}-\abs{d+c}$. From the Triangle Inequality, $\abs{d}+\abs{c}\geq\abs{d+c}\then\abs{d}+\abs{c}-\abs{d+c}\geq 0$ so we can drop the outermost absolute value signs and write $\abs{(\abs{a+b}+\abs{c}-\abs{a+b+c})}=\abs{a+b}+\abs{c}-\abs{a+b+c}$.

\item $\abs{x^2-2xy+y^2}$

$x^2-2xy+y^2=(x-y)^2\then \abs{x^2-2xy+y^2}=\abs{(x-y)^2}=(x-y)^2=x^2-2xy+y^2$.

\item $\abs{(\abs{\sqrt{2}+\sqrt{3}}-\abs{\sqrt{5}-\sqrt{7}})}$

Since $\sqrt{2}>0$ and $\sqrt{3}>0$, $\sqrt{2}+\sqrt{3}>0$ and so $\abs{\sqrt{2}+\sqrt{3}}=\sqrt{2}+\sqrt{3}$. Since $\sqrt{5}<\sqrt{7}$, $\sqrt{5}-\sqrt{7}<0$ so $\abs{\sqrt{5}-\sqrt{7}}=-(\sqrt{5}-\sqrt{7})=\sqrt{7}-\sqrt{5}$. Then $\abs{(\abs{\sqrt{2}+\sqrt{3}}-\abs{\sqrt{5}-\sqrt{7}})}=\abs{\sqrt{2}+\sqrt{3}+\sqrt{5}-\sqrt{7}}$.
\end{enumerate}

% Problem 10
\item Express each of the following without absolute value signs, treating various cases separately when necessary.
\begin{enumerate}[label=(\roman*)]
\item $\abs{a+b}-\abs{b}$

Case 1: $a+b\geq 0,b\geq 0\iff b>0,a\geq -b\then \abs{a+b}=a+b, \abs{b}=b\then\abs{a+b}-\abs{b}=a+b-b=a$.\\
Case 2: $a+b\leq 0,b\geq 0\iff b>0,a\leq -b\then \abs{a+b}=-(a+b), \abs{b}=b\then\abs{a+b}-\abs{b}=-(a+b)-b=-a-2b$.\\
Case 3: $a+b\geq 0,b\leq 0\iff b\leq 0,a\geq -b\then \abs{a+b}=a+b,\abs{b}=-b\then\abs{a+b}-\abs{b}=a+b+b=a+2b$.\\
Case 4: $a+b\leq 0,b\leq 0\iff b\leq 0,a\leq -b\then \abs{a+b}=-(a+b),\abs{b}=-b\then\abs{a+b}-\abs{b}=-(a+b)+b=-a$.

\item $\abs{({\abs{x}-1)}}$

Case 1: $x\geq 0\then \abs{x}=x\then\abs{({\abs{x}-1)}}=\abs{x-1}$. Case 1a $x-1\geq 0\iff x\geq 1\then \abs{x-1}=x-1$. Case 1b $x-1\leq 0\iff 0\leq x\leq 1\then \abs{x-1}=-(x-1)=1-x$.\\
Case 2: $x\leq 0\then\abs{x}=-x\then\abs{({\abs{x}-1)}}=\abs{-x-1}$. Case 2a $-x-1\geq 0\iff x\leq -1\then\abs{-x-1}=-x-1$. Case 2b $-x-1\leq 0\iff 0\geq x\geq -1\then\abs{-x-1}=-(-x-1)=x+1$.

Summarizing the cases: (1) $x\geq 1: x-1$, (2) $0\leq x \leq 1: 1-x$, (3) $-1\leq x \leq 0: x+1$, (4) $x\leq -1: -x-1$.

\item $\abs{x}-\abs{x^{2}}$

Case 1: $x\geq 0\then \abs{x}=x\then \abs{x}-\abs{x^{2}}=x-x^2$.\\
Case 2: $x\leq 0\then \abs{x}=-x\then \abs{x}-\abs{x^{2}}=-x-x^2$.

\item $a-\abs{(a-\abs{a})}$

Case 1: $a\geq 0\then\abs{a}=a\then a-\abs{(a-\abs{a})}=a-\abs{a-a}=a-0=a$.\\
Case 2: $a\leq 0\then\abs{a}=-a\then a-\abs{(a-\abs{a})}=a-\abs{a-(-a)}=a-\abs{2a}=a-(-2a)=3a$.
\end{enumerate}

% Problem 11
\item Find all numbers $x$ for which
\begin{enumerate}[label=(\roman*)]
\item $\abs{x-3}=8$.

$\abs{x-3}=x-3$ if $x-3>0\iff x>3$. For $x>3$, $x-3=8\then x=11$. $\abs{x-3}=-(x-3)$ for $x-3<0\iff x<3$. For $x<3$, $3-x=8\then x=-5$.
Solution: $x=11,-5$.

\item $\abs{x-3}<8$.

$\abs{x-3}=x-3$ if $x-3>0\iff x>3$. For $x>3$, $x-3<8\then x<11$. $\abs{x-3}=-(x-3)$ for $x-3<0\iff x<3$. For $x<3$, $3-x<8\then x>-5$.
Solution: $-5<x<11$.

\item $\abs{x+4}<2$.

$\abs{x+4}=x+4$ if $x+4>0\iff x>-4$. For $x>-4$, $x+4<2\then x<-2$. $\abs{x+4}=-(x+4)$ for $x+4<0\iff x<-4$. For $x<-4$, $-x-4<2\then x<-6$.
Solution: $x<-6$, $-4<x<-2$.

\item $\abs{x-1}+\abs{x-2}>1$.



\item $\abs{x-1}+\abs{x+1}<2$.



\item $\abs{x-1}+\abs{x+1}<1$.



\item $\abs{x-1}\cdot\abs{x+1}=0$.



\item $\abs{x-1}\cdot\abs{x+2}=3$.




\end{enumerate}


\end{document}